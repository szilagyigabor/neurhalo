% pdf/a 
\begin{filecontents*}[overwrite]{\jobname.xmpdata}
	\Title{Koncentrált paraméterű RF szűrő optimalizációja aktív tanulással}
	\Author{Pintér Bálint, Szilágyi Gábor}
	\Language{hu-HU}
	\Subject{Bayes-optimalizáció}
	\Keywords{Bayes-optimalizáció, aktív tanulás}
	\Publisher{Pintér Bálint, Szilágyi Gábor}
\end{filecontents*}

\documentclass[a4paper,12pt,titlepage]{article}
%\documentclass[a4paper,12pt,titlepage,draft]{article}
\usepackage{ucs}
\usepackage[T1]{fontenc}
\usepackage[utf8]{inputenc}
\usepackage[magyar]{babel}
\usepackage{amsfonts}
\usepackage{amsmath,bm}
%\usepackage{mathtools} % for \ceil*{}
\usepackage{amssymb}
\usepackage{graphicx}
%\usepackage[hang]{caption}
\usepackage{subcaption}
\usepackage{blkarray,booktabs,bigstrut} % a címkézett mátrixhoz
%\usepackage{enumerate}
%\usepackage{psfrag}
\usepackage[left=25mm,right=25mm,top=25mm,bottom=25mm]{geometry}
%\usepackage[hyphenbreaks]{breakurl}
%\usepackage[hyphens]{url}
%\usepackage{multirow}
%\usepackage{booktabs}
\usepackage{hyperref}
\usepackage{listings}
%\usepackage{cite}
%\usepackage{csquotes}
\usepackage{inconsolata}
\usepackage{siunitx}
\usepackage{xcolor}
\usepackage[a-3u]{pdfx}
\hypersetup{
	colorlinks,
%	 linkcolor={red!50!black},
	linkcolor={black},
%	 citecolor={blue!50!black},
	citecolor={black},
%	 urlcolor={blue!80!black}
	urlcolor={blue!80!black}
}

\definecolor{mygray}{RGB}{240, 240, 240}
\definecolor{mygreen}{RGB}{0, 140, 40}

\sisetup{
	range-phrase=--,
	range-units=single,
	output-decimal-marker={,},
	tight-spacing=true,
	print-unity-mantissa=false,
}

\lstset{ % General setup for the package
	language=Python,
	basicstyle=\scriptsize\ttfamily,
	numbers=left,
	numberstyle=\tiny,
	tabsize=2,
	backgroundcolor=\color{mygray},
	columns=fixed,
	showstringspaces=false,
	showtabs=false,
	keepspaces,
	frame=trbl,
	breaklines=true,
	%breakwhitespace=true,
	morekeywords={sort2,sort8},
	stringstyle=\color{red},
	commentstyle=\color{mygreen},
	keywordstyle=\color{blue}
}

\sloppy % Margón túllógó sorok tiltása.
\widowpenalty=10000 \clubpenalty=10000 %A fattyú- és árvasorok elkerülése
\def\hyph{-\penalty0\hskip0pt\relax} % Kötőjeles szavak elválasztásának engedélyezése

\begin{document}
\begin{center}
	\huge{Neurális hálózatok házi feladat beszámoló} \\
	\vspace*{0.5cm}
	\large{Pintér Bálint (I6QS0K), Szilágyi Gábor (NOMK01)}\\
	\vspace*{0.5cm}
	\Large{Koncentrált paraméterű RF szűrő optimalizációja \\ aktív tanulással}
\end{center}
\section{Bevezetés}
	\subsection{Aktív tanulás}
		A legtöbb neurális hálózatokat használó megoldás olyan problémára irányul, ahol sok rendelkezésre álló adat alapján kell a hálót betanítani egy feladat elvégzésére. Az a legtöbb esetben teljesül, hogy még több tanító adat felhasználásával jobb hálót lehetne tanítani, de ennek az extrém esetére tud megoldást nyújtani az aktív tanulás. A nem adathiányos problémáknál a rendelkezésre álló, címkézett adatpontok nagy részét felhasználva szokás tanítani a hálót, majd a fennmaradó adatpontokon ellenőrizni a háló teljesítőképességét olyan esetekre, amikkel nem talélkozott a tanulás során. Az aktív tanulás folyamata ettől merőben eltér.

		Aktív tanulás kiindulási helyzete, hogy nagyon sok címkézetlen adat áll rendelkezésre, de az egyes adatpontok címkézése rendkívül költséges. A címkézés kölstége miatt végeredményben az a cél, hogy azt minél kevesebbszer kelljen elvégezni a tanulás során. A tanulási folyamat közben az eddig megkapott kevés címkézett adatpont alapján a háló jelöl ki következőnek címkézésre azt, amelyik várhatóan a leghasznosabb lesz számára. A hasznosság becslésére több megközelítés is létezik, erre a későbbiekben visszetérünk.

		Az aktív tanulás egyik alesete a Bayes-optimalizáció. Itt nem egy osztályozót tanítunk minél kevesebb címkézett adat alapján, hanem egy ,,fekete doboz'' függvény maximumát keressük a függvény minél kevesebb kiértékelése mellett. Ez a különbség már befolyásolni fogja a következőnek megcímkézendő adat választását, ami ebben az esetben a következő paraméterértékek megválasztását jelenti, ahol kiértékeljük a függvényt.
	\subsection{Az optimalizálandó probléma}
		Az optimalizáció céljának 
	\subsection{A felhasznált könyvtár}
\section{A probléma átalakítása}
	\subsection{Impedanciák és admittanciák}
	\subsection{Láncparaméterek}
	\subsection{Szórási paraméterek}
\section{A célfüggvény}
	\subsection{Az első verzió és a problémái}
	\subsection{A paraméterek logaritmizálása}
	\subsection{A függvényérték logaritmizálása}
\clearpage
\appendix
\section{Az általunk írt specifikáció osztály}
	\lstinputlisting{../software/LC_opt_defs.py}
\end{document}
