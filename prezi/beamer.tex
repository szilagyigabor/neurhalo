\documentclass[aspectratio=169]{beamer}
\usepackage{ucs}
\usepackage[T1]{fontenc}
\usepackage[utf8]{inputenc}
\usepackage[magyar]{babel}
\usepackage{amsfonts}
\usepackage{amsmath,bm}
\usepackage{amssymb}
\usepackage{graphicx}
%\usepackage{blindtext}
%\usepackage[hang]{caption}
%\usepackage{subcaption}
%\usepackage{blkarray,booktabs,bigstrut} % a címkézett mátrixhoz
%\usepackage{enumerate}
%\usepackage{psfrag}
%\usepackage[left=25mm,right=25mm,top=25mm,bottom=25mm]{geometry}
%\usepackage[hyphenbreaks]{breakurl}
%\usepackage[hyphens]{url}
%\usepackage{multirow}
%\usepackage{booktabs}
%\usepackage{hyperref}
%\usepackage{listings}
%\usepackage{cite}
%\usepackage{csquotes}
\usepackage{circuitikz}
\usepackage{siunitx}
\usepackage{xcolor}
%\usepackage[a-3u]{pdfx}

%\definecolor{rosewood}{rgb}{0.6, 0.0, 0.04}
%\definecolor{indigo(dye)}{rgb}{0.0, 0.25, 0.42}

\usetheme{default}	% téma
\usefonttheme{serif}	% gyönyörű talpas betűtípus

\beamertemplatenavigationsymbolsempty % a pdf-be ágyazott navigációs gombok kikapcsolása
\newcommand\mat[1]{\underline{\underline{#1}}}

\title[Aktív tanulás, RF szűrő]{Koncentrált paraméterű RF szűrő optimalizációja\\aktív tanulással} % cím
\subtitle[]{} 	% alcím
\date{\today}
\author[P.B., Sz.G.]{Pintér Bálint, Szilágyi Gábor}	% szerző
\institute{BME VIK} % intézmény vagy más infó a szerzőről

\begin{document}
\maketitle	% címoldal
\begin{frame}
	\frametitle{Az optimalizálandó hálózat egysége}
	%\framesubtitle{Az 1. dia alcíme}
    \begin{center}
    	\begin{circuitikz}[] %[longL/.style = {cute inductor, inductors/scale=0.75, inductors/width=1.6, inductors/coils=9}]
            \draw (-0.5,0)
            to[short, o-*] (2,0)
            to[short, -*] (2,0.5)
            to[short, -] (1,0.5);
            \draw (1,2.5)
            to[C, l=$C_{i,p}$] (1,0.5);
            \draw (2,2.5)
            to[short, -] (3,2.5)
            to[L, l=$L_{i,p}$] (3,0.5)
            to[short, -] (2,0.5);
            \draw (-0.5,3)
            to[short, o-*] (2,3)
            to[short, -*] (2,2.5)
            to[short, -] (1,2.5);
            \draw (2,0)
            to[short, -o] (8,0);
            \draw (2,3)
            to[short, -*] (5,3)
            to[short, -] (5,2)
            to[L, a=$L_{i,s}$] (7,2)
            to[short, -] (7,3);
            \draw (5,3)
            to[short, -] (5,4)
            to[C, a=$C_{i,s}$] (7,4)
            to[short, -*] (7,3)
            to[short, -o] (8,3);
            \draw[thick,red] (0,-0.5) rectangle (7.5,5);
        \end{circuitikz}
    \end{center}
\end{frame}
\begin{frame}
	\frametitle{Az egységhálózat helyettesítőképe}
	%\framesubtitle{Az 1. dia alcíme}
    \begin{center}
    	\begin{circuitikz}[] %[longL/.style = {cute inductor, inductors/scale=0.75, inductors/width=1.6, inductors/coils=9}]
            \draw (-0.5,0)
            to[short, o-*] (2,0)
            to[short, -] (2,0.5);
            \draw (2,2.5)
            to[generic, l=$Z_{i,p}(f)$] (2,0.5);
            \draw (-0.5,3)
            to[short, o-*] (2,3)
            to[short, -] (2,2.5);
            \draw (2,0)
            to[short, -o] (8,0);
            \draw (2,3)
            to[short, -] (5,3)
            to[generic, a=$Z_{i,s}(f)$] (7,3)
            to[short, -o] (8,3);
            \draw[thick,red] (0,-0.5) rectangle (7.5,3.5);
        \end{circuitikz}
    \end{center}
\end{frame}
\begin{frame}
	\frametitle{Az egységhálózat szórási mátrixa (S-mátrixa)}
	%\framesubtitle{Az 1. dia alcíme}
    \begin{center}
    	\begin{circuitikz}[] %[longL/.style = {cute inductor, inductors/scale=0.75, inductors/width=1.6, inductors/coils=9}]
            \draw (-0.5,0)
            to[short, o-] (0,0);
            \draw (-0.5,3)
            to[short, o-] (0,3);
            \draw (7.5,0)
            to[short, -o] (8,0);
            \draw (7.5,3)
            to[short, -o] (8,3);
            \draw[thick,red] (0,-0.5) rectangle (7.5,3.5);
            \draw (3.75,1.5) node {$\mat{S_i}=\begin{bmatrix}S_{i,11}~S_{i,12} \\ S_{i,21}~S_{i,22}\end{bmatrix} \in \mathbb{C}^{2\times2}$};
        \end{circuitikz}
    \end{center}
\end{frame}
\begin{frame}
	\frametitle{Az optimalizálandó hálózat}
	%\framesubtitle{Az 1. dia alcíme}
    \begin{center}
    	\begin{circuitikz}[] %[longL/.style = {cute inductor, inductors/scale=0.75, inductors/width=1.6, inductors/coils=9}]
            \foreach \x in {1,...,5}
            {
                \draw (\x+\x-0.5,0)
                to[short, o-] (\x+\x+0,0);
                \draw (\x+\x-0.5,1)
                to[short, o-] (\x+\x+0,1);
                \draw[thick,red] (\x+\x+0,-0.2) rectangle (\x+\x+1,1.2);
                \draw (\x+\x+1,0)
                to[short, -o] (\x+\x+1.5,0);
                \draw (\x+\x+1,1)
                to[short, -o] (\x+\x+1.5,1);
                \draw (\x+\x+0.5,0.5) node {$\underline{\underline{S_{\x}}}$};
            }
        \end{circuitikz}
    \end{center}
\end{frame}
\begin{frame}
	\frametitle{Az optimalizálandó hálózat}
	%\framesubtitle{Az 1. dia alcíme}
    \begin{center}
    	\begin{circuitikz}[] %[longL/.style = {cute inductor, inductors/scale=0.75, inductors/width=1.6, inductors/coils=9}]
            \foreach \x in {1,...,5}
            {
                \draw (\x+\x-0.5,0)
                to[short, o-] (\x+\x+0,0);
                \draw (\x+\x-0.5,1)
                to[short, o-] (\x+\x+0,1);
                \draw[thick,red] (\x+\x+0,-0.2) rectangle (\x+\x+1,1.2);
                \draw (\x+\x+1,0)
                to[short, -o] (\x+\x+1.5,0);
                \draw (\x+\x+1,1)
                to[short, -o] (\x+\x+1.5,1);
                \draw (\x+\x+0.5,0.5) node {$\mat{S_{\x}}$};
            }
            \draw[thick,green] (1.75,-0.5) rectangle (11.25,1.5);    
        \end{circuitikz}
    \end{center}
    \begin{equation*}
        \mat{S} = \prod_{i}^{} \mat{S_{i}} = \begin{bmatrix}S_{11}~S_{12} \\ S_{21}~S_{22}\end{bmatrix}
    \end{equation*}
    \vspace{1cm}\\
    Az egész hálózat $|S_{21}(f)|$ paraméterére vonatkozik specifikáció.
\end{frame}
\begin{frame}
	\frametitle{A teljesítendő specifikáció}
	%\framesubtitle{Az 1. dia alcíme}
    asd
\end{frame}
\begin{frame}
	\frametitle{Az egydimenziós modell}
	\begin{columns}
		\column{0.48\textwidth}
			A következő egyszerűsítésekkel jutunk 3D-ből az 1D plazmához: \\
			\begin{itemize}
				\item A töltetlen $Xe$ részecskéket elhagyjuk
				\item Az ütközésektől eltekintünk
				\item Csak az elektronok mozgását vizsgáljuk
				\item Pontszerű részecskék helyett felületi töltéssűrűséggel rendelkező, az $x$ tengelyre merőleges lapok
				\item A pozitív töltésű $Xe^+$ ionokat helyhez kötött háttér-töltéssűrűségnek vesszük
			\end{itemize}
		\column{0.48\textwidth}
			\begin{itemize}
				\item A szimulációs tér egydimenziós és ciklikus, $x=0 \Longleftrightarrow x=N_g$
				\item A külső elektromos teret 0-nak vesszük
				\item A mégneses térnek nincs hatása 1D-ben
			\end{itemize}
	\end{columns}
\end{frame}
\begin{frame}
	\frametitle{A Particle-Mesh módszer}
	\begin{columns}
		\column{0.48\textwidth}
			Egy kis random szöveg
		\column{0.48\textwidth}
			\begin{figure}
				\includegraphics[width=\textwidth]{/home/g/Pictures/nap.png}
			\end{figure}
	\end{columns}
\end{frame}
\end{document}
